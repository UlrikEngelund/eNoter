
% = Farver
\definecolor{dturoed}{rgb}{0.6, 0.0, 0.0}
\definecolor{dtugraa}{rgb}{0.5, 0.5, 0.5}	% Lidt mørkere. Korrekt = 0.4
\definecolor{mingroenstreg}{rgb}{0.4,0.8,0}	% Sekundærfarve 14 : 102/204/0	(Forårsgrøn) -> Eksempler
\definecolor{mingroen}{rgb}{0.32,0.64,0}		% Sekundærfarve 14, 80% mørkere (tekst)
\definecolor{minorangestreg}{rgb}{1,0.6,0}		% Sekundærfarve 1 : 255/153/0	(Orange) -> Opgaver
\definecolor{minorange}{rgb}{0.8,0.48,0}		% Sekundærfarve 1 , 80% mørkere (tekst)

\definecolor{minblaa}{rgb}{0.2,0.4,0.8}	% Sekundærfarve 13 , 51/102/204 	( Blå -> Definitioner etc)
\definecolor{minmblaa}{rgb}{0.16,0.32,0.64}	% Sekundærfarve 13 , 80% mørkere (tekst)
\definecolor{thmbackground}{rgb}{0.97,.97, 0.99}	% Farve 13 - lys baggrund

\definecolor{mingraastreg}{rgb}{.5,.5,.5}
\definecolor{hvadbackground}{rgb}{0.97,.97, 0.97}
\definecolor{sumgul}{rgb}{1,1,.8}

\definecolor{hjmopgfarve}{rgb}{.96,1,.96}